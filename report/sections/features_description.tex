\section{Features description}
\label{sec:features}

In this section, we describe the features that we used in our experiment.

Some features are actually more significant than others, i.e. they provide a 
better information in order to tell if a tweet will be retweeted or not.
The more features we use, the more the computation becomes 
complex. This increased complexity is more significant with some classifiers 
than others but this means nevertheless that we need to circumscribe the number 
of our features. Furthermore, some features may provide such an irrelevant, or 
even non existent, increase of useful information that they shall just be 
removed.

In order to perform the feature selection for classification, various methods 
such as information gain (IG) and chi-square (CHI) can be used. We used the 
latter to determine which features were relevant for our case.

The table \ref{tab:features} lists all the features, order from the most 
significant to the less significant. Features that have been dropped are in 
italic font. The removal criterion was the $p-value>0.0$. When taking those 
features off, the accuracy increases by several points of percentage.

\begin{table}[!h]
 \centering
 \begin{tabular}{|c|l|c|c|}
  \hline
  \tabhead{\#} &
  \multicolumn{1}{|p{0.3\columnwidth}|}{\centering\tabhead{Features analyzed}} &
  \multicolumn{1}{|p{0.1\columnwidth}|}{\centering\tabhead{chi-square}} &
  \multicolumn{1}{|p{0.1\columnwidth}|}{\centering\tabhead{p-value}} \\
  \hline
  1  & \# of followers & 11332870 & 0.0\\
  2  & \# of tweets issued by the author & 1248000 & 0.0\\
  3  & \# of user's friends & 998714 & 0.0\\
  4  & is the tweet a retweet & 20615 & 0.0\\
  5  & \# of words in the tweet & 10095 & 0.0\\
  6  & \# of users mentioned & 5957 & 0.0\\
  7  & is the tweet a reply & 3030 & 0.0\\
  8  & \# of times "favorited" & 2007 & 0.0\\
  9  & has an URL & 790 & 0.0\\
  10 & \# of hashtags & 219 & 0.0\\
  11 & \textit{is the author a verified account} & 18 & 1.8e-05\\
  12 & \textit{tweet's age} & 16 & 7.7e-05\\
  13 & \textit{highest term frequency (TF)} & NaN & NaN\\
  14 & \textit{TF-IDF feature} & NaN & NaN\\
  \hline
 \end{tabular}
 \caption{This table lists the features used in our work.}
 \label{tab:features}
\end{table}

The most significant feature is the number of people that the user follows.
This value is collected from the \emph{User} object attached to a tweet.

The number of tweets (including retweets), apart from the one analyzed, issued 
by the author comes at the second place. This information is provided by the 
attribute "statuses\_count" of the \emph{User} object.

In third place is the number of friends (i.e. the number of users following 
the author). This value is provided by the \emph{User} object.

As said before, a tweet can be a retweet itself. Being at the fourth place, 
this feature is still sufficiently significant. This information was 
extracted from the text of the tweet which appears as a citation of the original 
tweet in the form "RT @username:".

The number of words in a tweet comes afterwards. The tweet's text was tokenized 
prior to count the number of words.

The feature in sixth position gives the number of users mentioned in a tweet.
This information is computed based on the list of "user\_mentions" given by the 
\emph{User} object.

The next feature determines whether a tweet is a reply to another tweet.This 
information is provided by the attribute "in\_reply\_to\_user\_id" 
which contains a non-null value if the tweet is a reply.

The feature number 8 corresponds to the number of times a tweet has been 
"favorited" by other users. This value is given by the attribute 
"favorite\_count" of a tweet object.

The following feature is based on the URLs found, or not, in a tweet 
The list of URLs is provided by the \emph{Tweet entities} object.

The number of hashtags feature is simply the number of hashtags found in a 
\emph{Tweet} object.

The remaining features have been disabled. 

The feature number 11 indicates if the author of the tweet has a verified 
account. Such accounts establish the authenticity of the user's identity. This 
feature would probably have been much more significant if the dataset was not 
limited by the theme of \textit{shampoo}.

The tweet's age, i.e. the number of days passed since the tweet was posted, is 
not surprisingly insignificant.

We were unfortunately not able to evaluate the significance of the term 
frequency (TF) and the the term frequency - inverse document frequency 
(TF-IDF). The chi-square obtained was $NaN$ and therefore the $p-value$ too.
Moreover, we were not able to compute the TF-IDF at all due to a bug in 
the \emph{scikit} library that we use. We are pretty disappointed by this fact 
because the TF and especially the TF-IDF features would probably have been very 
good candidates.

We would have loved to try other features. Among those we wanted to implement 
but did not because we were short on time were the author's account age, the 
number of tweets per day.

