\section{Features description}
\label{sec:features}

In this section, we describe the features that are relevant for our work and 
mention the other features that have been removed from the final experiment. The 
table \ref{tab:features} lists all the features.

\begin{table}[!h]
 \centering
 \begin{tabular}{|c|l|}
  \hline
  \tabhead{\#} &
  \multicolumn{1}{|p{0.7\columnwidth}|}{\centering\tabhead{Features used}} \\
  \hline
  1  & is the tweet a retweet \\
  2  & is the tweet a reply \\
  3  & number of followers \\
  4  & number of times "favorited" \\
  5  & number of tweets issued by the author \\
  6  & is the author a verified account \\
  7  & number of user's friends \\
  8  & number of hashtags \\
  9  & number of words in the tweet \\
  10 & number of users mentionned \\
  11 & has an URL \\
  12 & TF feature \\
  13 & tweet's age \\
  14 & TF-IDF feature \\
  \hline
 \end{tabular}
 \caption{This table lists the features used in our work. Their order has no 
  specific meaning.}
 \label{tab:features}
\end{table}

The first feature indicates if a tweet is a retweet or not. This information is 
extracted from the text of the tweet which appears as a citation of the 
original tweet with the form "RT @username:". 

The feature number 2 determines whether a tweet is a reply to another 
tweet.This information is provided by the attribute "in\_reply\_to\_user\_id" 
which contains a non-null value if the tweet is a reply.

The third feature counts the number of followers the author has. This value 
is given by the \emph{User} object attached to a tweet.

The feature number 4 indicates how many times a tweet has been "favorited" by 
other users. This count is given by the attribute "favorite\_count"  of a tweet 
object.

The fifth feature gives the number of other tweets (including retweets) issued 
by the author. This information is provided by the attribute "statuses\_count" 
of the \emph{User} object.

The feature number 6 indicates if the author of the tweet has a verified 
account. Such accounts establish the authenticity of the user's identity. In 
our case, this means that the author of the tweet is famous and then he will be 
probably retweeted.

The seventh feature is the number of friends (aka the number users following 
the author). This count is provided by the \emph{User} object. The number of 
friends is strongly related to the number of retweets.

The feature number 8 counts the number of hashtags from the \emph{Tweet} 
object and the feature number nine is the number of words of the tweet.

The tenth feature gives the number of users mentionned with a "@username" in 
the tweet. This information is calculated from the list of "user\_mentions" 
given by the \emph{User} object.

The feature number 11 indicates if the tweet contains an URL or not. The 
list of URL is provided by the \emph{Tweet entities} object.

Since the three last features did not give good results, they have been 
removed. The TF (Term Frequency) and the Tweet age were not significant while 
the TF-IDF (Term Frequency - Inverse Document Frequency) feature did not work 
due to a bug in the \emph{scikit} library.